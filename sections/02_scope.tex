\section{Scope and Applicability}

This document defines the scope and applicability of the Testing Automation Suite (TAS) as a shared testing capability within the Testing Center of Excellence.

\subsection{In-Scope Usage}

TAS is applicable to regression testing scenarios where system outputs are expected to remain stable across incremental changes, unless an intentional change has been introduced. In-scope usage includes scenarios requiring one or more of the following capabilities:

\begin{itemize}
  \item Comparison of system outputs across successive executions, versions, or states.
  \item Validation of calculated metrics, derived attributes, and aggregated results.
  \item Identification and localisation of unintended differences introduced through system changes.
  \item Support for regression testing across multiple systems, domains, or platforms using a consistent testing approach.
\end{itemize}

These scenarios commonly arise in high-impact systems, including but not limited to finance, risk, regulatory reporting, and other critical business processes.

\subsection{Conditional Usage}

TAS may be applied selectively in the following situations:

\begin{itemize}
  \item Early-stage development or exploratory testing where no stable reference state exists.
  \item One-off analytical investigations not intended to support formal test outcomes.
  \item Prototyping activities where testing intent is still evolving.
\end{itemize}

In such cases, TAS functions primarily as an accelerator rather than a reference testing capability.

\subsection{Out-of-Scope Activities}

TAS does not attempt to automate or replace:

\begin{itemize}
  \item Interpretation of business, policy, or regulatory intent.
  \item Design decisions regarding expected system behaviour.
  \item Exploratory testing aimed at discovering previously unknown behaviours.
\end{itemize}

These activities may inform testing inputs but remain outside the scope of TAS.

\subsection{Applicability Constraint}

Where TAS is applied as the reference regression testing approach, outputs and discrepancies identified by TAS are treated as the baseline for further investigation and decision-making. Alternative analysis techniques may be used to provide explanation or context, but not to invalidate observed results.
